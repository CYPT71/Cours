\documentclass{article}
\title{Projet 1SGBD}
\author{Cyprien Taib, Lounes Behloul, Soufian Oualla}
\date{February 2021}


\begin{document}

    \maketitle
    \section{introduction}
        
    In this document we will see how we proced the task and why we make the golang chose for programming, after that we will speak about the tools we use.


    The projet is on our github in two part, first part is purlly the backend and maybe in the future the front end in two separates files. \newline\newline

    The seconde part is functionnal, it permit to create table, fill it and run it. We have also put a docker file to create the mysql server and PhpMyAdmin.
   
    
    \section{Backend Part}

    We choose golang for the backend because is more efficency and fastest than python. 

    We enter in the Backend file witch contains files and folders.

    The .air.toml we see is a config file for auto reload the API.

    The config.yaml is the config file for the API.

    The file go.mod and go.sum is for the golang modules.

    We see a directory named internal.

        In internal

    we see three directories
        \begin{enumerate}
            \item http
            \item sql request
            \item utils
        \end{enumerate}

    the utils is for loadding configuration into the server \newline\newline

    the folder http contains two directories and one file " http.go ". The file is like the main file for running the API.  

    The folder middlewares contain securty constrain and controllers all the handler for the API. \newline\newline

    Back to the sql request folder, it contains just all the code for making request to the data base. \newline\newline

    Each tables have it handler and it sql request file. \newline
    
    \section{Request Possibillity}

    \begin{itemize}
        \item List of company aircraft
        \item List of Sup Air Line pilots
        \item List of personnel by category
        \item List of passengers per flight
        \item List of flights to a given city
        \item List of departures for the day
        \item List of cities served by Sup Air Line
        \item List of destinations served by a captain
        \item List of pilots whose license must be renewed
        \item Lists of regular passengers who fly more than 2 flights / month
        \item Professions with the most regular passengers
        \item Number of hours worked by a captain
        \item Number of flight hours of each aircraft
        \item Number of passengers transported by plane over a given period
        \item Number of passengers carried over a given period
        \item Number of tickets sold per day / week / month
        \item Total sales
        \item Average flights per pilot
        \item Most profitable destinations (high occupancy rate)
        \item Average occupancy rate by plane / flight / destination
        \item Which pilots fly to their city
        \item Create Data for all tables
        \item Update Data for all tables
        \item Delete Data for all tables 
    \end{itemize}

    \section{Reuqest Details} 
    always will be path method description and ":variable" is change part of url
    \begin{itemize}
        \item /cabincrew 
                \begin{itemize}
                    \item / get the list of CabinCrew
                    \item / post add a new CabinCrew  
                    \item / path Update a CabinCrew
                    \item /:name delete delete a CabinCrew
                \end{itemize}
        \item /departures
            \begin{itemize}
                \item / get the list of departures
                \item / post add a new departures  
                \item / path Update a departures
                \item /:name delete delete a departures
            \end{itemize}
        \item /devices
            \begin{itemize}
                \item / get the list of device
                \item /time the time flight for device
                \item / post add a new device  
                \item / path Update a device
                \item /:name delete delete a device
            \end{itemize}
        \item /employees
            \begin{itemize}
                \item / get the list of employee
                \item /categories get the list of employee order by categories Pilote, CabinCrew, GroundStaff
                \item /categories the time flight for employee
                \item / post add a new employee  
                \item / path Update a employee
                \item /:name delete delete a employee
            \end{itemize}
        \item /flights
            \begin{itemize}
                \item / get the list of flight
                \item /:city get the list of a given city
                \item /categories the time flight for flight
                \item / post add a new flight  
                \item / path Update a flight
                \item /:name delete delete a flight
            \end{itemize}
        \item /passengers
            \begin{itemize}
                \item / get the list of passenger List
                \item /regular get the most regular Profession
                \item /perFlight get the passenger per flight
                \item /mostRegular get get Passenger Most regular per Flight
                \item /byPlane/:start/:end get Number of passenger by plane by period (Y-m-d)
                \item /total/:start/:end Number of passenger by period
                \item / post add a new passenger  
                \item / path Update a passenger
                \item /:name delete delete a passenger
            \end{itemize}       
        \item /pilote
            \begin{itemize}
                \item / get the list of pilote
                \item /details get details list about pilote
                \item /renew get the pilote who need to renew there licence
                \item /:name/:firstName get a list of pilote for a given name
                \item /flightHours get a list of flig hours for pilote
                \item /piloteArrival get pilote who go to there home
                \item /average get Pilote flight average
                \item / post add a new pilote  
                \item / path Update a pilote
                \item /:name delete delete a pilote
            \end{itemize}
        \item /route
            \begin{itemize}
                \item / get the list of route
                \item / post add a new route  
                \item / path Update a route
                \item /:name delete delete a route
            \end{itemize}
        \item /tickets
            \begin{itemize}
                \item / get the list of tickets
                \item /total get the total Sales of ticket
                \item /:interval get ticket sold filter by month, week, day
                \item / post add a new tickets  
                \item / path Update a tickets
                \item / delete delete a tickets
            \end{itemize}
    \end{itemize}

    \section{Data Generator}
    
    In order to set data in our database, we have made the choice to code a python script that will allow us to fill our database with more diversified data. \newline

    That would enable us both to make the output of the requests more interesting and to save an incredible amount of time. We thought that it more interesting to realize this generator. \newline
    
    In this code, a number of coding techniques were employed to make the most realistic and consistent dataset possible. You can have a look at our code for more information.


\end{document}